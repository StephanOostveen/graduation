\section{Research description}
\label{sec:researchdescription}
In Section~\ref{sec:domain} we have discussed the transition to new electrical/electronic architectures in the automotive domain and how Time Sensitive Networking will serve a core role in these architectures. Section~\ref{sec:problemstatement} explained that the transition to new architectures and technologies has associated risks and costs which need to be minimized for a successful transition. 
The numerous network layouts and configurations combined with the difficulty of evaluating the network performance of a specific configuration increase the risks and costs of a such a transition. Accurate estimates of network performance under various situations at an early stage in the design process will on the other hand decrease the risks and costs. In practice, the performance of multiple configurations must be evaluated before choosing which one shall be implemented. Stated differently, Lightyear is searching for an answer to the following question:

\begin{quote}
    \emph{What is an effective method for evaluating the effect of various network configurations and architectures based on Time Sensitive Networking at an early stage of development of an in-vehicle network?}
\end{quote}

The remainder of the section is structured as follows: Section~\ref{sec:researchquestion} describes the research question motivating the research performed in the graduation project and its relevance to answering the main question posed by Lightyear. The research contributions of the graduation project are laid out in Section~\ref{sec:researchcontribution} and a plan for developing these contributions is defined in Section~\ref{sec:researchdevelopment}. The risks associated with that plan are finally described in Section~\ref{sec:researchrisks}.

\subsection{Research question}
\label{sec:researchquestion}
Performing \textit{measurements} on the system is not very suitable for guiding the design of the system itself as it means you first need to build (a prototype of) the system. As a result the feedback cycle between designing a system and performing measurements on an instance of that system to guide the design is large. 

\textit{Analysis} of a system has the advantage that it can happen during the design phase, reducing the feedback cycle and thus the associated costs and risks. Analysis is also exact, resulting in high confidence in the performance of the system. Unfortunately accurate analysis can be difficult, the system might violate the assumptions of the available methods, and for certain configurations of Time Sensitive Networking the analysis methods are still under development~\cite{ashjaei2021time}. 

The final option is to get performance numbers through \textit{simulation} of the system. The advantages of simulation are that it does not depend on the availability of a real system and is not constrained by boundary conditions or assumptions about the system an analysis method might have. A disadvantage of simulation is that by definition the results are not exact and care should be taken when interpreting the results, especially when stochastic processes are used in the simulation. Due to the disadvantages of the measurement and analysis methods the graduation project will focus on creating a simulation based method to answer Lightyear's question. Within the domain of network simulations, the discrete event simulation model is popular~\cite{ashjaei2021time}. Examples of existing discrete event simulation frameworks are \omnet, NS3 and the Matlab Simulink Discrete-Event simulation toolbox. With \omnet being the popular framework in the research of simulating Time Sensitive Networks we propose a graduation project researching the following question:

\begin{quote}
\emph{How can simulation using the \omnet based INET framework guide the design of in-vehicle networks based on Time Sensitive Networking?}
\end{quote}

\subsection{Research contribution}
\label{sec:researchcontribution}
A simulation method for guiding the design of a TSN network at leasts needs the ability to model relevant network TSN standards and configurations, the ability to measure relevant performance indicators, and it should scale to the size of real-world networks and network traffic. The contribution of the proposed graduation project will give insights into the following questions:
\begin{itemize}
    \item Which TSN standards and configurations are relevant for an in-vehicle network?
    \item Which performance indicators guide the design of an in-vehicle network?
    \item What does a realistic automotive network and its traffic look like?

    % \item What are some of the network configurations that an architect wants to play with?
    % \item What are some of the parameters that should be measured in simulation to determine the quality of a network?
    % \item What analysis should be performed on the simulation results to guide decisions for specific network configurations.
    % \item What does a realistic automotive network and its traffic look like?
\end{itemize}

Time Sensitive Networking is a big group of standards that can be combined and configured in different ways. It is important to understand which networking problems the automotive industry wants to solve or avoid as it guides the choice in standards and configuration. Understanding the needs of the industry guides academia into development of relevant analysis and simulation techniques which are applicable in the industry. Similarly, insights in the performance indicators guiding the development of a network can lead towards new optimization techniques relevant for the industry.

Traditionally synthetic benchmarks are used in academia to evaluate the performance of proposed solutions as freely available descriptions of real-world automotive networks do not exist~\cite{ashjaei2021time}. One benchmark~\cite{kramer2015real} is available describing the characteristics of an internal combustion engine control application. Unfortunately this benchmark focusses on the real-time aspects of an application but does not describe the network architecture or traffic of that application. Furthermore, it is not clear how applicable the benchmark is to the system level of a vehicle. And lastly it is conceivable that an electric vehicle has different timing characteristics, different network architecture and dataflow than an internal combustion engine vehicle. For these reasons it is interesting to create a benchmark of a current generation electric vehicle's network.

A model will be proposed for simulating automotive in-vehicle networks based on the Time Sensitive Networking technologies. Taking the uncovered needs of the automotive industry into account such that it can be used to guide the development of new in-vehicle networks.

\subsection{Development approach}
\label{sec:researchdevelopment}
In this section we will briefly lay out the plan for finding an answer to the research question. To start, a machine-readable network and dataflow description will be created of the periodic data for both the logical view as the concrete deployment found in LY 0. The logical view is interesting because it describes the data requirements on the system level without dictating a concrete deployment. It can be used to generate different deployments implementing the same functionality. The result can be used during the development and validation of the simulation framework. Depending on the availability of time, the dataflow description can be augmented with a description of sporadic data and a definition of uses cases in which aperiodic data is communicated.

Secondly we want to start developing the simulation framework which we will start by investigating which features are required from the network and which problems the industry wants to solve using Time Sensitive Networking. With this information suitable solutions will be sketched for the required features and problems. For example, data streams that are involved in safety functions, such as braking, can require redundant transmission from source to end node. One solution would be to use 802.1CB (frame replication and elimination for reliability) for those streams, which is a static redundancy method. In contrast, software defined networking is a dynamic redundancy method that monitors the state of the network and changes the routing of data streams requiring redundancy at run-time when required, an approach is presented in~\cite{kong2021run}. Knowing the requirements and problems of the automotive industry we will research and list the available alternatives.

Thirdly, we will investigate what the guiding performance indicators are for an automotive in-vehicle network, as they are the desired outputs of the simulations guiding the design of in-vehicle networks. As mentioned in Section~\ref{sec:problemstatement} the schedule of the Time Aware Shaper can influence the performance of the network, for example using Optimization Modulo Theories a schedule can be generated which maximizes the Quality Of Service for best-effort traffic~\cite{houtan2021synthesising}. Strategies for optimizing the relevant performance indicators will be researched and methods for obtaining results from the simulations can be developed. 

Based on the list of relevant network standards, configurations and relevant performance indicators we can develop suitable interfaces and abstractions for defining and simulating a network in \omnet. The network and dataflow description of the Lightyear 0 will serve as input for simulating and evaluating the performance of different network configurations deemed relevant for the automotive industry.

\subsection{Risks}
\label{sec:researchrisks}
Several risks were identified for a successful completion of the proposed graduation project, we set out to answer the following questions in the feasibility experiments. Regarding the creation of a network and dataflow benchmark from the Lightyear 0's architecture:
\begin{itemize}
\item Is a complete definition of the in-vehicle network available in a machine or human-readable format?
\item What information is available on the network traffic in either a machine or human-readable format?
\end{itemize}

Regarding the use of \omnet as a simulation framework:
\begin{itemize}
\item Is the simulation framework \omnet actively maintained and well documented, i.e, is it "suitable" for use in an industrial setting?
\item What interfaces are provided by \omnet to define and run experiments?
\end{itemize}