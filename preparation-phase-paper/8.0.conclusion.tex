\section{Conclusion}
\label{sec:conclusion}
The automotive industry is transitioning to new electrical/electronics architectures. The core of this new architecture is a high speed switched network, the industry has selected Time Sensitive Networking (TSN) as the technology which enables real-time communication on a converged Ethernet based network. Transitioning to a new architecture has associated costs and risks which should be minimized for the transition to be successful. Part of the risk in the TSN technology is that the design space is large and the impact of decisions in network topology and configuration on the performance is non-trivial to understand. Delaying the performance evaluation to the test phase has high risks and thus the industry needs methods for evaluating the effect of various network configuration and architectures at an early stage of development. We have seen that the state-of-the-art analysis methods are still in development and do not always fit the desired network architecture. Simulation on the other hand is feasible at this moment due to the availability of Time Sensitive Networking models for the \omnet simulation framework. 

We propose a graduation project that investigates how simulation using the \omnet framework can guide the development of in-vehicle networks based on TSN. The contributions of the project are: insights in the relevance of TSN standards and configurations for an in-vehicle network. Secondly which performance indicators guide the design of an in-vehicle network. And lastly a benchmark describing a realistic current generation automotive network and the data transferred on this network for a solar electric vehicle. With this research we hope to give insights in the problems the industry is facing to guide further research in the automotive TSN domain.