\section{Problem statement}
\label{sec:problemstatement}
As mentioned in Section~\ref{sec:automotive-arch} the decentralized electrical/electronic architecture of modern vehicles has several drawbacks such as reliability, cost and software complexity. In addition, the weight reduction offered by a zonal architecture is important for Lightyear as it improves the vehicle efficiency. Using a new architecture can be difficult and costly as previous knowledge, assumptions and best practices should be reevaluated for the new architecture. Transitioning to a new architecture has certain associated risks and costs related to understanding the details and getting the system to work properly. For example, if a fundamental mistake in the architecture, such as non-determinism of transmission delays, is only discovered at the implementation verification phase of the project, the project might be delayed or cancelled and cost can increase significantly.

Time Sensitive Networking has been selected by the automotive industry as the technology upon which the zonal architecture will be build~\cite{klaus2019zonal}. But as mentioned in Section~\ref{sec:tsn} TSN is not a single standard but rather a set of standards that can be combined and configured in many ways to create a specific network capable of real-time frame transmission. For example the schedule used in the Time Aware Shaper influences performance metrics of the scheduled network traffic such as the worst-case delay and jitter, but also the average case throughput of lower priority classes and best-effort traffic. Various scheduling algorithms exist which compute offline transmission schedules for TSN, each algorithm has a different optimization goal and thus the network performance metrics change between scheduling algorithms. For example the scheduler presented in~\cite{steiner2010evaluation} uses Satisfiability Modulo Theories (SMT) to find a schedule which ensures that relevant constraints in the network are met (end-to-end delays, maximum switch queue size etc.). While the scheduling algorithm presented in~\cite{houtan2021synthesising} uses Optimization Modulo Theories (OMT) to find the schedule which satisfies the timing constraints of the scheduled traffic and maximizes the Quality of Service (QoS) for best-effort traffic. Not only the configuration of the TSN standards, but the physical network layout will impact those performance metrics. It is conceivable that Lightyear would want to optimize the physical network for some cost function containing price, weight and energy consumption, while still maintaining the strict timing requirements of the real-time traffic. 

For all the reasons mentioned above, development cost and risk, the numerous combinations and configuration options of the TSN standards and all the different physical network topologies, it is necessary to have (estimates of) performance numbers as early as possible in the development cycle. Having a short feedback cycle between design decision and (estimates of) performance numbers can improve product quality, decrease development costs and risks. By modelling the in-vehicle network and the embedded software using the services of the network, system architects can evaluate the performance of the network during the architecture design phase.