\subsection{Research contribution}
\label{sec:research_contribution}
Generating a model from source code and documentation describing the whole in-vehicle network and its traffic is a big task as there are a lot of components running code in the network. In most projects there will be some architecture describing how components interact, and most components will behave according to that main architecture. The core ideas of that architecture can be replicated in a model template describing the system. We will create a method to automatically collect the necessary data from the source code to instantiate that template. This will result in an approximation of the real system. Manual improvements can then be made to that generated model as to improve the accuracy of the model, e.g. by reading documentation or adding subsystems that are missed by the automatic generation process. Reasons for the process to miss certain details are: complexity of reverse engineering details from code in an automated way, bugs in the process, parts of the system which do not conform to the core architecture etc. The process can be used in similar applications where one wishes to (semi-)automatically generate a model from source code and documentation.

As mentioned in Section~\ref{sec:sota} not much is known about the traffic of an in-vehicle network as this is seen as a company secret by many manufacturers. The model being extracted from source code won't be an exact description but an approximation of the Lightyear 0's network traffic. Still an approximation of a realistic automotive network can be more useful in research than a random generated traffic profile, as it represents a real-life usage pattern. 

Finally, a study with OMNeT++ will show whether the extracted model is abstract enough to make simulation feasible and detailed enough to adequately answer performance questions at the system level.
