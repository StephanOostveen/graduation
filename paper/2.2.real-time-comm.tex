\subsection{Real-time communication over Ethernet}
\label{sec:real-time-comm}
Some vehicle functions pose strict timing requirements on the architectures and implementations. For example the passenger safety system must always deploy the airbags within a specific time period after a crash. Deploying the airbags too quickly or too late can cause harm to the passengers. Similarly, when the driver presses the brake pedal in an electric vehicle the car should always stop accelerating, apply regenerative braking and illuminate the brake lights within a well-defined period of time. Large or non-deterministic variations in response time are undesirable from a safety and user experience point of view. Such systems with strict timing requirements are called real-time systems and require the timing properties to be bounded and analysable. 

Traditional automotive networks such as LIN, CAN and FlexRay support real-time communication, while IEEE 802.3 Ethernet is not a real-time network. One of the original design philosophies is best-effort transmission of frames~\cite{metcalfe1976ethernet}. Putting the responsibility of dealing with packet loss, duplication, large delays etc. on the communicating processes. This gives good average-case performance to a large group of users in an economically viable way, but sacrifices the real-time property.

The switches used in a full-duplex switched Ethernet network introduce a delay in the communication between two nodes. IEEE 802.3 does not specify how a switch should operate, hence in general no guarantee can be made about the queuing and switching delay added to the transmission time by a switch. Additionally, it is reasonable to assume that queuing delays increase as the number of packets passing through a switch increases. Making Ethernet unsuitable for use in a real-time system. 

Several higher level protocols have been proposed to make real-time communication on top of Ethernet possible e.g EtherCAT, PROFINET, TTEthernet and Time Sensitive Networking. Some of these protocols require special network interface cards or switches to operate, or they are not directly interoperable with \textit{standard} Ethernet, which complicates mixing the network other devices such as a computer using the TCP/IP stack.

Time Sensitive Networking (TSN) is a set of standards created by an IEEE Task Group. According to their website\footnote{\url{https://1.ieee802.org/tsn/}, accessed 15 December 2023} their goal is "\textit{... to provide deterministic connectivity through IEEE 802 networks, i.e., guaranteed packet transport with bounded latency, low packet delay variation and low packet loss.}" TSN standards and amendments to standards can be grouped in four categories according to their design goal~\cite{ashjaei2021time}: Timing and synchronization, resource management, bounded low latency and finally high reliability. For each design goal several solutions have been standardized, for example several traffic shapers exist which aim for bounded low latency. The traffic shapers are designed to solve different problems, potentially improving the network performance, but complicating the network design as many choices need to be made during the design phase. Making the design more complex is the fact that most standards can be combined in the same network. For example the Credit Based Shaper~\cite{IEEE8021Qav} and Time Aware Shaper~\cite{IEEE8021Qbv} can be combined. The relevant standards and amendments for the automotive industry have been summarized in Appendix~\ref{appendix:tsn}.