\subsection{Controller Area Network}
\label{sec:can}
Controller Area Network (CAN) is a networking technology developed by Robert Bosch GMbH in 1983 and later standardized in 1993 as ISO 11898:1993. CAN is an asynchronous serial communication protocol with data rates up to 1 Mbit/s specifically designed for real-time control~\cite{ISO11898} and is organized as a bus architecture. Meaning that each node on the network may transmit a message when it wants, as it is a bus architecture a transmitted message is received by every node on the bus. Carrier Sense Multiple Access/Collision Resolution is used to avoid multiple nodes transmitting concurrently resulting in data errors. The CAN Frame layout together with the physical encoding of the data bits ensures that when two frames are transmitted at the same time the highest priority frame will be delivered. The sender of the lower priority frame detects the collision before data loss occurs and stops transmission, automatically retrying transmission after the higher priority frame is fully delivered. 