\section{Research description}
\label{sec:research_description}
In Section~\ref{sec:domain} we discussed the transition to new electrical/electronics architectures in the automotive industry which are based on zone ECUs and a centralized control system. In this new architecture Time Sensitive Networking will be the main network technology for the future generation of vehicles. Section~\ref{sec:problem_statement} explained that the transition to a new architecture based on Time Sensitive Networking is non-trivial. We concluded that a solution capable of estimating a network's performance during the design process is necessary to reduce the risks of delay or cancellation of a project. The two most relevant solutions for determining network performance in the design phase are mathematical analysis and simulation. Various analysis methods exist but have a limited applicability due to the assumptions made. Specifically the implementation of the CAN bridge in the Lightyear 0 makes analysis of the network performance complex. Considering the popularity and availability of CAN and TSN models for \omnet, the decision was made to focus the research on an \omnet based simulation. In section~\ref{subsec:benchmark} we have seen that the available vehicle benchmarks either miss relevant data or do not fairly represent the Lightyear 0. For this reason we will research a method capable of extracting a model from source code that is abstract enough to make simulation feasible, but detailed enough to adequately answer performance questions at the system level. The remainder of this section is structured as follows: in section~\ref{sec:research_question} we refine Lightyear's problem from Section~\ref{sec:problem_statement}. Section~\ref{sec:research_contribution} describes the researcher's contribution to the problem of performance evaluation.

\subsection{Research question}
\label{sec:research_question}
As mentioned in section~\ref{sec:problem_statement} transitioning from a CAN based architecture to a Time Sensitive Networking based architecture can be difficult and costly. An approach to reduce the risks is to minimize the amount of changes inside a project. In this case, changing the network stack, hardware and physical network to the Time Sensitive Networking technology while maintaining the original \textit{application} software.

To successfully transition to the Time Sensitive Networking technology Lightyear must understand the requirements its software is posing on the CAN based in-vehicle network as this will be the same for the Time Sensitive Networking network. The topology of the current network together with the traffic size, periodicity, priority, source and destination gives an indication of those minimal requirements. Understanding the application characteristics will guide the decision of the specific network configuration, for example which traffic shaper to use and how to allocate the available bandwidth. The \textit{Lightyear 0} can serve as a typical example of a battery electric vehicle, as it has a CAN based decentralized architecture.

Regarding the method, knowledge of the performance metrics which guide the network design, i.e. those that make a network \textit{good} or \textit{bad}, together with knowledge of the subsystems affecting these metrics will inform the method's design. Gaining insight in the effect of a certain network configuration on the key performance metrics is the main goal of the method as the outcome dictates whether a certain configuration needs to be altered or is good enough. Knowing the subsystems that affect the network performance means that we can design the method such that these subsystems can be altered easily, enabling fast experimentation.

As mentioned in section~\ref{sec:sota} three methods for evaluating network performance exist: measurements, analysis and simulation. For the reasons mentioned in that section we will limit ourselves to a simulation based approach. Specifically the OMNeT++ framework as it is popular in the field of network simulations and has models available for both Time Sensitive Networking and CAN based networks, making it a good choice for the research. Taking the previous into consideration we come to the following research questions:

\begin{quote}
    \emph{How can we create a model of the Lightyear 0 from source code and documentation in order to characterize the in-vehicle network its traffic.}
\end{quote}

\begin{quote}
    \emph{How can OMNeT++ be used to experiment with various network configurations at a vehicle level?}
\end{quote}
\subsection{Research contribution}
\label{sec:research_contribution}
Generating a model from source code and documentation describing the whole in-vehicle network and its traffic is a big task as there are a lot of components running code in the network. In most projects there will be some architecture describing how components interact, and most components will behave according to that main architecture. The core ideas of that architecture can be replicated in a model template describing the system. We will create a method to automatically collect the necessary data from the source code to instantiate that template. This will result in an approximation of the real system. Manual improvements can then be made to that generated model as to improve the accuracy of the model, e.g. by reading documentation or adding subsystems that are missed by the automatic generation process. Reasons for the process to miss certain details are: complexity of reverse engineering details from code in an automated way, bugs in the process, parts of the system which do not conform to the core architecture etc. The process can be used in similar applications where one wishes to (semi-)automatically generate a model from source code and documentation.

As mentioned in Section~\ref{sec:sota} not much is known about the traffic of an in-vehicle network as this is seen as a company secret by many manufacturers. The model being extracted from source code won't be an exact description but an approximation of the Lightyear 0's network traffic. Still an approximation of a realistic automotive network can be more useful in research than a random generated traffic profile, as it represents a real-life usage pattern. 

Finally, a study with OMNeT++ will show whether the extracted model is abstract enough to make simulation feasible and detailed enough to adequately answer performance questions at the system level.
