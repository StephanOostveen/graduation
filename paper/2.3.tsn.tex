\subsection{Time Sensitive Networking}
\label{sec:tsn}
Time Sensitive Networking (TSN) is a set of standards created by an IEEE Task Group. According to their website\footnote{\url{https://1.ieee802.org/tsn/}, accessed 15 December 2023} their goal is "\textit{... to provide deterministic connectivity through IEEE 802 networks, i.e., guaranteed packet transport with bounded latency, low packet delay variation and low packet loss.}" TSN standards and amendments to standards can be grouped in four categories according to their design goal~\cite{ashjaei2021time}: Timing and synchronization, resource management, bounded low latency and finally high reliability. For each design goal several solutions have been standardized, for example several traffic shapers exist which aim for bounded low latency. The traffic shapers are designed to solve different problems, potentially improving the network performance, but complicating the network design as many choices need to be made during the design phase. Making the design more complex is the fact that most standards can be combined in the same network. For example the Credit Based Shaper~\cite{IEEE8021Qav} and Time Aware Shaper~\cite{IEEE8021Qbv} can be combined. The relevant standards and amendments for the automotive industry have been summarized in Appendix~\ref{appendix:tsn}.