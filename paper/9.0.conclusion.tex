\section{Conclusion}
\label{sec:conclusion}
This research started because Lightyear wanted to investigate how Time Sensitive Networking (TSN) could be applied in a solar vehicle, following the trend in automotive architectures. As we have seen in section~\ref{sec:real-time-comm} and appendix~\ref{appendix:tsn} TSN is not a single standard like CAN, but a set of standards which can be combined and configured in different ways resulting in vastly different performance. Combined with the fact that all prior knowledge of the CAN based architectures cannot be used it is unlikely that choosing a network configuration without some form of analysis will lead to success. To reduce complexity further it was assumed that existing applications would be reused, and only the network interface software would be changed. Hence the research focussed on finding a method for evaluating the effect of various network architectures assuming code reuse. Three methods for performance exist, performing measurements on a system, mathematical analysis and simulation. The problem with measurements is that one first needs to build the system, which is impractical for Lightyear. As we have seen in section~\ref{subsec:analysis} analysis methods exist for TSN and CAN, the advantages are that it gives clear and exact insights in the relationship between variables and performance. Drawbacks are that reality must often be simplified to make it feasible and specifically in the case of TSN the methods are still under development. Simulation is not an exact method, but the system can be modelled more accurately. Simulation also allows inspecting the inner working of the subsystems without interfering with the results as is the case with measurements. Given the availability of CAN and TSN models for the \omnet simulation framework it was chosen as a basis for the evaluation method. In section~\ref{subsec:benchmark} we looked at several available benchmarks but concluded that they did not fairly represent the Lightyear 0's in-vehicle network. Which is necessary for the simulation to be able to evaluate the effects of the network on Lightyear's software.

We have shown the call-graph analysis method is capable of automatically extracting data from source code for the creation of a communication model at the application level. This method was possible due to the specific source code interfaces used in the Lightyear 0's software. Implementing a similar strategy in different projects would allow the same method to be applied. The method is relatively easy to implement as the call-graph generation is implemented by the freely available LLVM compilation toolchain. A drawback of the method is that it overestimates the actual data transfer since it lacks context to find conditional data transfers. As a side effect the resulting model can be used to analyse the communication which is actually implemented and compare it with architectural designs as demonstrated in section~\ref{sec:benchmark}. We found several interesting cases that should be investigated by a system designer, e.g. 162 data dictionaries that are either never read or never produced but still require system resources. 

In section~\ref{sec:experiments} some experiments where performed to demonstrate that the extracted model is detailed enough to find bottlenecks and evaluate network performance of the CAN based network. The model is capable of simulating 8 seconds per real second on five-year-old consumer hardware and does not require large amounts of memory, allowing it to be executed by developers on their normal machines. Our experiments regarding the data dictionary age demonstrate that the model is detailed enough to evaluate the effects of a network configuration on the performance of the applications.

Because the runnables communicate with each other in terms of data dictionaries, the lower level communication layers can be interchanged transparently. Meaning the physical transport of data between the programmable end nodes can be changed from CAN to TSN without requiring to change the runnables or RTOS in the simulation. This matches the automotive industry's strategy for the transition to a zonal architecture. And our model was developed with this in mind. The CAN transmission can be adapted to TSN using the freely available TSN models from the INET moddel collection. A new mapping for data dictionaries to Ethernet frames must be developed and a suitable alternative for the powertrain and vehicle can busses must be investigated. Finally, assumptions on undocumented parts of the system where made and for various system parameters such as bit error rates. Analysis of a suitable model for these parameters must be performed before the simulation results can be used for performance analysis. 