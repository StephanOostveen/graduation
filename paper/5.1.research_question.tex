\subsection{Research question}
\label{sec:research_question}
As mentioned in section~\ref{sec:problem_statement} transitioning from a CAN based architecture to a Time Sensitive Networking based architecture can be difficult and costly. An approach to reduce the risks is to minimize the amount of changes inside a project. In this case, changing the network stack, hardware and physical network to the Time Sensitive Networking technology while maintaining the original \textit{application} software.

To successfully transition to the Time Sensitive Networking technology Lightyear must understand the requirements its software is posing on the CAN based in-vehicle network as this will be the same for the Time Sensitive Networking network. The topology of the current network together with the traffic size, periodicity, priority, source and destination gives an indication of those minimal requirements. Understanding the application characteristics will guide the decision of the specific network configuration, for example which traffic shaper to use and how to allocate the available bandwidth. The \textit{Lightyear 0} can serve as a typical example of a battery electric vehicle, as it has a CAN based decentralized architecture.

Regarding the method, knowledge of the performance metrics which guide the network design, i.e. those that make a network \textit{good} or \textit{bad}, together with knowledge of the subsystems affecting these metrics will inform the method's design. Gaining insight in the effect of a certain network configuration on the key performance metrics is the main goal of the method as the outcome dictates whether a certain configuration needs to be altered or is good enough. Knowing the subsystems that affect the network performance means that we can design the method such that these subsystems can be altered easily, enabling fast experimentation.

As mentioned in section~\ref{sec:sota} three methods for evaluating network performance exist: measurements, analysis and simulation. For the reasons mentioned in that section we will limit ourselves to a simulation based approach. Specifically the OMNeT++ framework as it is popular in the field of network simulations and has models available for both Time Sensitive Networking and CAN based networks, making it a good choice for the research. Taking the previous into consideration we come to the following research questions:

\begin{quote}
    \emph{How can we create a model of the Lightyear 0 from source code and documentation in order to characterize the in-vehicle network its traffic.}
\end{quote}

\begin{quote}
    \emph{How can OMNeT++ be used to experiment with various network configurations at a vehicle level?}
\end{quote}