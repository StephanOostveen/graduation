\section{Problem statement}
\label{sec:problem_statement}
As mentioned in Section~\ref{sec:architectures} the decentralized architecture of current vehicles has several drawbacks such as the low available bandwidth, reliability, cost and software complexity. The centralized zonal architectures aim to solve these problems and can potentially reduce the power usage and weight of the vehicle as there are less ECUs and cables in the vehicle, making the zonal architecture interesting for Lightyear.

Transitioning from a well understood architecture and network to a new architecture and network technology is a difficult and costly task as all prior knowledge, assumptions, best practices and tools become obsolete and need to be developed from scratch. Errors in the architecture which are discovered late in the project may cause the project to be delayed or cancelled, increasing costs significantly. The chance of making a mistake while designing the Time Sensitive Network is significant. As mentioned in Section~\ref{sec:tsn} Time Sensitive Networking is not a single technology but rather a large set of independent standards which can be mixed to achieve a certain design goal. Each independent part of the network can be configured in different ways with different outcomes. For example the schedule of the Time Aware Shaper influences worst-case delay and jitter of messages in the network. But various algorithms exist for generating that schedule, each with a different optimization goal and thus different performance. The physical network layout impacts the performance characteristics as well, since a message that needs to traverse more switches will incur more delay.

Because of the complexity of Time Sensitive Networking a solution that is capable to estimate a network's performance would allow designers to iterate over various network designs. Gaining confidence that the final network will behave as expected and meet the relevant requirements such as message delay and jitter. A strategy to reduce the complexity of the transition between architectures is to reuse the embedded software executing the vehicle functions while only changing the network software. Taking all this into consideration, Lightyear is searching for an answer to the following question:

\begin{quote}
    \emph{What is an effective method for evaluating the effect of various network configurations and architectures based on Time Sensitive Networking when reusing software from a CAN based architecture?}
\end{quote}

\todo{dit kan iets uitgebreider}