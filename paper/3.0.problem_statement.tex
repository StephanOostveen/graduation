\section{Problem statement}
\label{sec:problem_statement}
As mentioned in Section~\ref{sec:architectures} the decentralized architecture of current vehicles has several drawbacks such as the low available bandwidth, reliability, cost and software complexity. The centralized zonal architectures aim to solve these problems and can potentially reduce the power usage and weight of the vehicle as there are less ECUs and cables in the vehicle, making the zonal architecture interesting for Lightyear. Transitioning to a new architecture and network technology is a difficult and costly task as all prior knowledge, assumptions, best practices and tools become obsolete and need to be redeveloped. Errors in the architecture which are discovered late in the project may cause the project to be delayed or cancelled, increasing costs significantly. As mentioned in Section~\ref{sec:real-time-comm} Time Sensitive Networking is not a single technology but rather a large set of independent standards that can be mixed to achieve a certain design goal. Each independent part of the network can be configured in different ways with different outcomes. For example the schedule of the Time Aware Shaper influences worst-case delay and jitter of messages in the network. But various algorithms for generating the schedule exist, each with a different optimization goal and thus different performance. The physical network layout impacts the performance characteristics as well, since a message that needs to traverse more switches will incur more delay. Taking this into consideration, designing a Time Sensitive Network with adequate performance is a non-trivial effort.

Because of the complexity of Time Sensitive Networking a solution capable of estimating a network's performance would allow designers to iterate over various network designs. Gaining confidence that the final network will perform as expected and meet the relevant requirements posed by the applications. The complexity of adopting a zonal architecture can be reduced by using a two-step process. The first step focusses on implementing a TSN backbone between the main ECUs in the network, changes to the applications should be kept to a minimum. In this stage the required knowledge and tools for working with TSN are built up. In the second stage the zonal architecture is implemented fully, function based ECUs are consolidated into zonal ECUs and a centralized control architecture is implemented. Taking all this into consideration, Lightyear is searching for an answer to the following question:

\begin{quote}
    \emph{What is an effective method for evaluating the effect of various network configurations and architectures based on Time Sensitive Networking when reusing software from a CAN based architecture?}
\end{quote}

Three methods exist for quantifying a system's performance: measurements on a real system, mathematical analysis and simulation. Measurements are precise in determining a system's performance as you are measuring the real system with all its implementation details. It is unsuited for determining the relationship between variables and performance as experimentation can take a long time and measuring the inner workings of the system without affecting the system is hard. One first needs to build the system before measurements can be made, making it unsuitable for a design process. Mathematical analysis can be performed without building a system, gives clear and exact insights in the relationship between variables and performance. Analysis can be complex and reality must often be simplified to make it feasible. In certain cases the simplified system differs significantly from the real system, making it questionable whether the results are applicable to the real system. Simulation mixes the properties of measurements and analysis. It is still not an exact method for determining a system's performance as not all details will be modelled. The models can better represent the analysed system compared to the models used in analysis. Crucially it allows inspecting the inner workings of the subsystems without interfering with the result as is the case with measurements. Giving insight in the relationships between variables and performance. 

Part of Lightyear's question is the wish to reuse application software, meaning changes to the communication characteristics should stay minimal. The evaluation method should be tailored to the network traffic of the Lightyear 0. To evaluate the performance of a network a description of the communication requirements is necessary. For example, message description, periods and network topology. It is worth investigating whether in-vehicle network benchmarks resembling the Lightyear 0 are available that can be used as input for the analysis.