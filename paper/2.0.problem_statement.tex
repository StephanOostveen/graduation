\section{Problem statement}
As mentioned in Section~\ref{sec:automotive-arch} the decentralized electrical/electronic architecture of modern vehicles has several drawbacks such as reliability, cost and software complexity. In addition, the weight reduction offered by a zonal architecture is important for Lightyear as it improves the vehicle efficiency. Using a new architecture can be difficult and costly as previous knowledge, assumptions and best practices should be reevaluated for the new architecture. Transitioning to a new architecture has certain associated risks and costs related to understanding the details and getting the system to work properly. For example, if a fundamental mistake in the architecture, such as non-determinism of transmission delays, is only discovered at the implementation verification phase of the project, the project might be delayed or cancelled and cost can increase significantly.

Time Sensitive Networking has been selected by the automotive industry as the technology upon which the zonal architecture will be build~\cite{klaus2019zonal}. But as mentioned in Section~\ref{sec:tsn} TSN is not a single standard but rather a set of standards that can be combined and configured in many ways to create a specific network capable of real-time frame transmission. For example the schedule used in the Time Aware Shaper influences performance metrics of the scheduled network traffic such as the worst-case delay and jitter, but also the average case throughput of lower priority classes and best-effort traffic. Various scheduling algorithms exist which compute offline transmission schedules for TSN, each algorithm has a different optimization goal and thus the network performance metrics change between scheduling algorithms. For example the scheduler presented in~\cit{steiner2010evaluation} uses Satisfiability Modulo Theories (SMT) to find a schedule which ensures that relevant constraints in the network are met (end-to-end delays, maximum switch queue size etc.). While the scheduling algorithm presented in~\cite{houtan2021synthesising} uses Optimization Modulo Theories (OMT) to find the schedule which satisfies the timing constraints of the 


Laten zien dat er veel opties zijn waar keuzes voor gemaakt moeten worden met verschillende effecten (scheduler generators bijvoorbeeld).

Beschrijf dat voor een goed automotive product je verschillende analyses wilt uitvoeren zoals timing analysis om te verifieren of je aan alle timing specificaties voldoet. Maar ook voor functional safety in het geval van bijvoorbeeld link failure. Het is risicovol om deze analyses pas uit te voeren tijdens de implemenetatie / validatie fase van een project omdat je dan niet meer op tijd problemen kunt oplossen. Als je een tool hebt om op basis van een architectuur beschrijving de performance van je architectuur uit te rekenen dan kun je met enige mate van zekerheid zeggen dat een implemenetatie gaat werken. Bijkomend voordeel is dat je gegeven genoeg tijd de performance van verschillende potentiele architecturen uit kunt rekenen en dus gegronde keuzes kunt maken.
