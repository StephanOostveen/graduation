\section{Approach}
\label{sec:approach}
We start by describing the nodes found in the Lightyear 0's network and how they are networked in section~\ref{subsec:network}. Followed by an in-depth description of the software execution model of the nodes developed by Lightyear. Notably we find that the functionality implemented by the tasks is decoupled from the physical deployment. The tasks communicate in terms of data dictionaries as opposed to CAN messages. The relationship between tasks can be described in terms of data dictionaries, which is more interesting than CAN messages. The data dictionary view makes it possible to experiment with different deployments and different mappings to network packets, while observing the effects on the applications. Section~\ref{subsec:moddeling} describes the implementation in \omnet, the differences compared to the vehicle's implementation and the assumptions made in cases where documentation was lacking. In section~\ref{subsec:model_gen} we develop a method based on call-graph analysis to retrieve from source code the data required for performing simulation experiments on the network.

\subsection{Lightyear 0's embedded system}
In this section we describe which networks exist in the Lightyear 0, which nodes are connected to which network and the inner workings of the networked nodes. This information is retrieved through meetings with developers, reading documentation from Lightyear and suppliers and reading the Lightyear 0's source code.
\paragraph{In vehicle network}
\label{subsec:network}
The \textit{Lightyear 0's} in-vehicle network is based on a decentralized functional architecture, with CAN and LIN as the main network technologies. Ethernet is used for non-real time applications such as infotainment, over-the-air updates and debugging. This means that networked components pertaining to the same function or domain (body control, drivetrain etc.) are grouped together in a network. The networks are linked together through a \textit{Central Gateway} which bridges the relevant data between networks. For example the \textit{motor power} might be required in the \textit{battery management system}, \textit{propulsion controller} and \textit{media controller}. But for various reasons these controllers might not all be connected to the same network, so the \textit{motor power} needs to be forwarded on the relevant networks.

At the core of the embedded system are three ECUs named the \textit{Central Gateway (CGW)}, \textit{Safety Control Unit (SCU)}, \textit{Vehicle Control Unit (VCU)}. The \textit{Safety Control Unit} is an off the shelf \textit{OpenECU M560} which contains two heterogeneous cores, that both have access to the connected CAN busses. The \textit{Vehicle Control Unit} is also an \textit{OpenECU M560}, but the secondary core is not used, while the \textit{Central Gateway} uses a \textit{OpenECU M130}. Each ECU has a real-time operating system with a fixed priority preemptive scheduler and can be connected up to four CAN busses and two LIN busses. While the RTOS is provided by the manufacturer of the \textit{OpenECUs} the software running inside the RTOS is developed by Lightyear and thus the functionality of an ECU is not fixed. For this reason we will call these ECUs \textit{programmable end nodes} for modelling purposes.

The other nodes in the network have a fixed function, some of them are developed by Lightyear such as motor inverters and solar converters, while others are developed by third parties. Some of these can be configured by the user to tune the behaviour, while others are completely fixed. For modelling purposes we call these nodes \textit{parametrizable end nodes}. The inner workings of these nodes varies and is unknown in case of a third party component. Fortunately for network simulation only the external behaviour needs to be known. Because these nodes must be integrated, a precise specification of the network traffic generated and consumed should be available and thus can be used when modelling a node.

From Lightyear's internal documentation we were able to generate an overview of all the networks in the car and which nodes connect to which network. The two LIN busses, LIN A and LIN B, each connect to the \textit{Central Gateway} which acts as the master in the LIN networks. The slave nodes are third party of the shelf components and can be classified as \textit{parametrizable end nodes}. Table~\ref{tab:networkoverview} represents a partial overview of the nodes, the networks and their connections in the Lightyear 0. Unfortunately Lightyear's documentation was not complete and retrieving the relevant information of every node would have taken a significant effort. Due to time reasons a subset of the nodes mentioned in the table have been modelled, those have been coloured green. An alternative schematic overview can be found in Figure~\ref{fig:networkoverview}.

Finally, there are components in the vehicle that generate or consume data but are not seen as a networked component in the "traditional" sense. Examples are speakers, rearview cameras and instrument cluster displays. Further investigation is necessary to determine how many of these nodes exist, how they are connected and how they should be modelled from a networking perspective.
\begingroup
\renewcommand*{\arraystretch}{1}
\begin{table}[htb]
    \centering
    \resizebox{\textwidth}{!}{%
    % \begin{tabular}{@{}lllllllllllllllllll@{}}
    \begin{tabular}{lllllllllllllllllll}
                                & \multicolumn{10}{c}{Controller Area Network} & \multicolumn{2}{c|}{LIN} & \multicolumn{5}{c}{Ethernet} & \multicolumn{1}{c}{Other} \\* \cmidrule(lr){2-11} \cmidrule(r){12-13} \cmidrule(r){14-18} \cmidrule(r){19-19}
    Node name                & \multicolumn{1}{R{2.5cm}}{Driver support} & \multicolumn{1}{R{2cm}}{Drivetrain} & \multicolumn{1}{R{2cm}}{Energy\\ management} & \multicolumn{1}{R{2cm}}{HVBS} & \multicolumn{1}{R{2cm}}{Powertrain} & \multicolumn{1}{R{2cm}}{Solar} & \multicolumn{1}{R{2cm}}{Surrounding\\ sense} & \multicolumn{1}{R{2cm}}{Telematics} & \multicolumn{1}{R{2cm}}{Vehicle} & \multicolumn{1}{R{2cm}}{Vehicle 2} & \multicolumn{1}{R{2cm}}{LIN A} & \multicolumn{1}{R{2cm}|}{LIN B} & \multicolumn{1}{R{2cm}}{Media} & \multicolumn{1}{R{2cm}}{Inverter FL} & \multicolumn{1}{R{2cm}}{Inverter FR} & \multicolumn{1}{R{2cm}}{Inverter RL} & \multicolumn{1}{R{2cm}}{Inverter RR} & \multicolumn{1}{R{2cm}}{Others}   \\*\cmidrule(r){1-1} \cmidrule(r){2-11}\cmidrule(r){12-13} \cmidrule(r){14-18} \cmidrule(r){19-19}
    \textcolor{mint}{Vehicle control unit}       & X &   &   &   &   & X & X &   & X &   &   & \multicolumn{1}{c|}{}  &   &   &   &   &   &   \\
    \rowcolor[gray]{0.925}
    \textcolor{mint}{Central gateway}            &   &   &   &   & X &   &   & X & X & X & X & \multicolumn{1}{c|}{X} &   &   &   &   &   &   \\
    \textcolor{mint}{Safety control unit}        &   & X & X & X & X &   &   &   &   &   &   & \multicolumn{1}{c|}{}  &   &   &   &   &   &   \\
    \rowcolor[gray]{0.925}
    \textcolor{mint}{Steering column module}     & X &   &   &   &   &   &   &   &   &   &   & \multicolumn{1}{c|}{}  &   &   &   &   &   &   \\
    \textcolor{mint}{Inverter FL}                &   & X &   &   &   &   &   &   &   &   &   & \multicolumn{1}{c|}{}  &   & X &   &   &   &   \\
    \rowcolor[gray]{0.925}
    \textcolor{mint}{Inverter FR}                &   & X &   &   &   &   &   &   &   &   &   & \multicolumn{1}{c|}{}  &   &   & X &   &   &   \\
    \textcolor{mint}{Inverter RL}                &   & X &   &   &   &   &   &   &   &   &   & \multicolumn{1}{c|}{}  &   &   &   & X &   &   \\
    \rowcolor[gray]{0.925}
    \textcolor{mint}{Inverter RR}                &   & X &   &   &   &   &   &   &   &   &   & \multicolumn{1}{c|}{}  &   &   &   &   & X &   \\
    High voltage battery                         &   &   &   & X &   &   &   &   &   &   &   & \multicolumn{1}{c|}{}  &   &   &   &   &   &   \\
    \rowcolor[gray]{0.925}
    On-board charger                             &   &   & X &   &   &   &   &   &   &   &   & \multicolumn{1}{c|}{}  &   &   &   &   &   &   \\
    \textcolor{mint}{Media ECU}                  &   &   &   &   & X &   &   &   &   &   &   & \multicolumn{1}{c|}{}  & X &   &   &   &   & X \\
    \rowcolor[gray]{0.925}
    Steering angle sensor                        &   &   &   &   & X &   &   &   &   &   &   & \multicolumn{1}{c|}{}  &   &   &   &   &   &   \\
    \textcolor{mint}{Dual string controller 1}   &   &   &   &   &   & X &   &   &   &   &   & \multicolumn{1}{c|}{}  &   &   &   &   &   &   \\
    \rowcolor[gray]{0.925}
    \textcolor{mint}{Dual string controller 2}   &   &   &   &   &   & X &   &   &   &   &   & \multicolumn{1}{c|}{}  &   &   &   &   &   &   \\
    \textcolor{mint}{Dual string controller 3}   &   &   &   &   &   & X &   &   &   &   &   & \multicolumn{1}{c|}{}  &   &   &   &   &   &   \\
    \rowcolor[gray]{0.925}
    \textcolor{mint}{Dual string controller 4}   &   &   &   &   &   & X &   &   &   &   &   & \multicolumn{1}{c|}{}  &   &   &   &   &   &   \\
    \textcolor{mint}{Dual string controller 5}   &   &   &   &   &   & X &   &   &   &   &   & \multicolumn{1}{c|}{}  &   &   &   &   &   &   \\
    \rowcolor[gray]{0.925}
    \textcolor{mint}{Dual string controller 6}   &   &   &   &   &   & X &   &   &   &   &   & \multicolumn{1}{c|}{}  &   &   &   &   &   &   \\
    \textcolor{mint}{Dual string controller 7}   &   &   &   &   &   & X &   &   &   &   &   & \multicolumn{1}{c|}{}  &   &   &   &   &   &   \\
    \rowcolor[gray]{0.925}
    \textcolor{mint}{Dual string controller 8}   &   &   &   &   &   & X &   &   &   &   &   & \multicolumn{1}{c|}{}  &   &   &   &   &   &   \\
    \textcolor{mint}{Dual string controller 9}   &   &   &   &   &   & X &   &   &   &   &   & \multicolumn{1}{c|}{}  &   &   &   &   &   &   \\
    \rowcolor[gray]{0.925}
    \textcolor{mint}{Dual string controller 10}  &   &   &   &   &   & X &   &   &   &   &   & \multicolumn{1}{c|}{}  &   &   &   &   &   &   \\
    \textcolor{mint}{Dual string controller 11}  &   &   &   &   &   & X &   &   &   &   &   & \multicolumn{1}{c|}{}  &   &   &   &   &   &   \\
    \rowcolor[gray]{0.925}
    \textcolor{mint}{Dual string controller 12}  &   &   &   &   &   & X &   &   &   &   &   & \multicolumn{1}{c|}{}  &   &   &   &   &   &   \\
    \textcolor{mint}{Dual string controller 13}  &   &   &   &   &   & X &   &   &   &   &   & \multicolumn{1}{c|}{}  &   &   &   &   &   &   \\
    \rowcolor[gray]{0.925}
    \textcolor{mint}{Dual string controller 14}  &   &   &   &   &   & X &   &   &   &   &   & \multicolumn{1}{c|}{}  &   &   &   &   &   &   \\
    Camera monitoring system                     &   &   &   &   &   &   & X &   &   &   &   & \multicolumn{1}{c|}{}  &   &   &   &   &   &   \\
    \rowcolor[gray]{0.925}
    \textcolor{mint}{Parking sensor 1}           &   &   &   &   &   &   & X &   &   &   &   & \multicolumn{1}{c|}{}  &   &   &   &   &   &   \\
    \textcolor{mint}{Parking sensor 2}           &   &   &   &   &   &   & X &   &   &   &   & \multicolumn{1}{c|}{}  &   &   &   &   &   &   \\
    \rowcolor[gray]{0.925}
    \textcolor{mint}{Parking sensor 3}           &   &   &   &   &   &   & X &   &   &   &   & \multicolumn{1}{c|}{}  &   &   &   &   &   &   \\
    \textcolor{mint}{Parking sensor 4}           &   &   &   &   &   &   & X &   &   &   &   & \multicolumn{1}{c|}{}  &   &   &   &   &   &   \\
    \rowcolor[gray]{0.925}
    \textcolor{mint}{Parking sensor 5}           &   &   &   &   &   &   & X &   &   &   &   & \multicolumn{1}{c|}{}  &   &   &   &   &   &   \\
    \textcolor{mint}{Parking sensor 6}           &   &   &   &   &   &   & X &   &   &   &   & \multicolumn{1}{c|}{}  &   &   &   &   &   &   \\
    \rowcolor[gray]{0.925}
    \textcolor{mint}{Parking sensor 7}           &   &   &   &   &   &   & X &   &   &   &   & \multicolumn{1}{c|}{}  &   &   &   &   &   &   \\
    \textcolor{mint}{Parking sensor 8}           &   &   &   &   &   &   & X &   &   &   &   & \multicolumn{1}{c|}{}  &   &   &   &   &   &   \\
    \rowcolor[gray]{0.925}
    Telematics control unit                      &   &   &   &   &   &   &   & X &   &   &   & \multicolumn{1}{c|}{}  & X &   &   &   &   &   \\
    Window wiper                                 &   &   &   &   &   &   &   &   & X &   &   & \multicolumn{1}{c|}{}  &   &   &   &   &   &   \\
    \rowcolor[gray]{0.925}
    \textcolor{mint}{RC compressor}              &   &   &   &   &   &   &   &   & X &   &   & \multicolumn{1}{c|}{}  &   &   &   &   &   &   \\
    \textcolor{mint}{Tailgate latch}             &   &   &   &   &   &   &   &   &   & X &   & \multicolumn{1}{c|}{}  &   &   &   &   &   &   \\
    \rowcolor[gray]{0.925}
    \textcolor{mint}{User authentication system} &   &   &   &   &   &   &   &   &   & X &   & \multicolumn{1}{c|}{}  &   &   &   &   &   &   \\
    \textcolor{mint}{User authentication app}    &   &   &   &   &   &   &   & X &   &   &   & \multicolumn{1}{c|}{}  &   &   &   &   &   &   \\
    \rowcolor[gray]{0.925}
    \textcolor{mint}{Alarm system}               &   &   &   &   & X &   &   &   &   &   &   & \multicolumn{1}{c|}{}  &   &   &   &   &   &   \\
    \textcolor{mint}{Front left door latch}      &   &   &   &   & X &   &   &   &   &   &   & \multicolumn{1}{c|}{}  &   &   &   &   &   &   \\
    \rowcolor[gray]{0.925}
    \textcolor{mint}{Front right door latch}     &   &   &   &   &   &   &   &   &   & X &   & \multicolumn{1}{c|}{}  &   &   &   &   &   &   \\
    \textcolor{mint}{Rear left door latch}       &   &   &   &   &   &   &   &   &   & X &   & \multicolumn{1}{c|}{}  &   &   &   &   &   &   \\
    \rowcolor[gray]{0.925}
    \textcolor{mint}{Rear right door latch}      &   &   &   &   & X &   &   &   &   &   &   & \multicolumn{1}{c|}{}  &   &   &   &   &   &   \\
    \textcolor{mint}{Filling tool}               &   &   &   &   &   &   &   &   & X &   &   & \multicolumn{1}{c|}{}  &   &   &   &   &   &   \\
    \rowcolor[gray]{0.925}
    Rain light sensor                            &   &   &   &   &   &   &   &   &   &   & X & \multicolumn{1}{c|}{}  &   &   &   &   &   &   \\
    Air flap actuator                            &   &   &   &   &   &   &   &   &   &   &   & \multicolumn{1}{c|}{X} &   &   &   &   &   &   \\
    \rowcolor[gray]{0.925}
    Speaker left                                 &   &   &   &   &   &   &   &   &   &   &   & \multicolumn{1}{c|}{}  &   &   &   &   &   & X \\
    Display                                      &   &   &   &   &   &   &   &   &   &   &   & \multicolumn{1}{c|}{}  &   &   &   &   &   & X \\
    \rowcolor[gray]{0.925}
    Rearview camera                              &   &   &   &   &   &   &   &   &   &   &   & \multicolumn{1}{c|}{}  &   &   &   &   &   & X \\
\end{tabular}%
}
\caption{Partial overview of in-vehicle networks and nodes}
\label{tab:networkoverview}
\end{table}
\endgroup

\begin{figure}[htb]
    \centering
\begin{tikzpicture}[
    programmablenode/.style={rectangle, draw=lightblue, fill=lightblue},
    parametrizablenode/.style={rectangle, draw=orange, fill=orange},
    canbus/.style={rectangle, fill=mint},
    linbus/.style={rectangle, fill=pink},
    ethernet/.style={rectangle, fill=olive},
    other/.style={rectangle, fill=palegrey},
]
    % Nodes & Networks
    \node[programmablenode] (SCU) {Safety Control Unit};
    \node[canbus] (PT) [above=of SCU] {Powertrain};
    \node[programmablenode] (CGW) [above=of PT]{Central Gateway};
    \node[canbus] (VEHICLE) [above=of CGW] {Vehicle};
    \node[programmablenode] (VCU) [above=of VEHICLE]{Vehicle Control Unit};
    \draw[-] (SCU.north) -- (PT.south);
    \draw[-] (CGW.south) -- (PT.north);
    % Driver support
    \node[canbus] (DS) [right= of VCU] {Driver Support};
    \node[parametrizablenode] (SCM) [above=of DS] {Steering Column Module};
    \draw[-] (DS.west) -- (VCU.east);
    \draw[-] (DS.north) -- (SCM.south);
    % Drivetrain
    \node[canbus] (DT) [below= of SCU] {Drivetrain};  
    \node[parametrizablenode] (INVFL) [below left=of DT]{Inverter FL};
    \node[parametrizablenode] (INVFR) [below =of INVFL]{Inverter FR};
    \node[parametrizablenode] (INVRL) at (DT|-INVFL){Inverter RL};
    \node[parametrizablenode] (INVRR) at (DS|-INVFL){Inverter RR};
    \draw[-] (SCU.south) -- (DT.north);
    \draw[-] (INVFL.north) -- (DT.west);
    \draw[-] (DT.south west) -- (INVFR.north east);
    \draw[-] (DT.south) -- (INVRL.north);
    \draw[-] (DT.south east) -- (INVRR.north west);
    % Energy Management
    \node[canbus] (EM) at (DS|-SCU) {Energy Managment};
    \node[parametrizablenode] (OBC) [below= of EM] {On-board charger};
    \draw[-] (EM.west) -- (SCU.east);
    \draw[-] (EM.south) -- (OBC.north);
    % Powertrain
    \node[parametrizablenode] (MEDIA) [left= of SCU] {Media ECU};
    \node[parametrizablenode] (SAS) at (DS|-PT) {Steering angle sensor};
    \draw[-] (MEDIA.north) -- (PT.west);
    \draw[-] (SAS.west) -- (PT.east);
    % Surounding Sense
    \node[canbus] (SS) [above= of VCU] {Surounding Sense};
    \node[parametrizablenode] (PSS) [above right=of SS] {Parking Sensor System};
    \node[parametrizablenode] (CMS) [above=of SS] {Camera Monitoring System};
    \draw[-] (VCU.north) -- (SS.south);
    \draw[-] (SS.north) -- (CMS.south);
    \draw[-] (SS.north east) -- (PSS.south west);
    % Solar
    \node[canbus] (SOLAR) [left=of SS] {Solar};
    \node[parametrizablenode] (DSC) [left=of CMS] {Dual String Converter 1};
    \draw[-] (SOLAR.south east) -- (VCU.north west);
    \draw[-] (DSC.south) -- (SOLAR.north);
    % Telematics
    \node[canbus] (TELE) [left=of PT] {Telematics};
    \node[parametrizablenode] (TCU) [left=of TELE] {Telematics Control Unit};
    \node[ethernet] (MEDIAETH) at(TCU|-MEDIA) {Media Ethernet};
    \draw[-] (CGW.south west) -- (TELE.north east);
    \draw[-] (TELE.west) -- (TCU.east);
    \draw[-] (MEDIAETH.north) -- (TCU.south);
    \draw[-] (MEDIAETH.east) -- (MEDIA.west);
    % HVBS
    \node[canbus] (HVBS) [below left= of SCU] {HVBS};
    \node[parametrizablenode] (BAT) at(MEDIAETH|-HVBS) {High Voltage Battery};
    \draw[-] (HVBS.north east) -- (SCU.south west);
    \draw[-] (HVBS.west) -- (BAT.east);
    % Vehicle
    \node[parametrizablenode] (WIP) at (DS|-VEHICLE) {Window Wiper};
    \node[parametrizablenode] (COMP) [below=of WIP] {RC Compressor};
    \draw[-] (CGW.north) -- (VEHICLE.south);
    \draw[-] (VEHICLE.north) -- (VCU.south);
    \draw[-] (VEHICLE.east) -- (WIP.west);
    \draw[-] (VEHICLE.south east) -- (COMP.west);
    % Vehicle2
    \node[canbus] (VEHICLE2) [left=of VCU] {Vehicle 2};
    \node[parametrizablenode] (LATCH) at (DSC|-SOLAR) {Tailgate Latch};
    \draw[-] (VCU.west) -- (VEHICLE2.east);
    \draw[-] (VEHICLE2.north west) -- (LATCH.south);
    %INVerter Ethernet
    \node[ethernet] (ETHFL) at(MEDIAETH|-INVFL) {Inverter FL};
    \draw[-] (INVFL.west) -- (ETHFL.east);
    \node[ethernet] (ETHFR) at(MEDIAETH|-INVFR) {Inverter FR};
    \draw[-] (INVFR.west) -- (ETHFR.east);
    \node[ethernet] (ETHRL) at(INVRL|-INVFR) {Inverter RL};
    \draw[-] (INVRL.south) -- (ETHRL.north);
    \node[ethernet] (ETHRR) at(INVRR|-INVFR) {Inverter RR};
    \draw[-] (INVRR.south) -- (ETHRR.north);
    %LIN busses
    \node[linbus] (LINA) at(MEDIA|-VEHICLE) {LINA};
    \node[parametrizablenode] (RAIN) at(TCU|-LINA) {Rain light sensor};
    \draw[-] (RAIN.east) -- (LINA.west);
    \draw[-] (LINA.east) -- (CGW.north west);
    \node[linbus] (LINB) at(MEDIA|-CGW) {LINB};
    \node[parametrizablenode] (AIR) at (TCU|-LINB) {Air flap actuator};
    \draw[-] (AIR.east) -- (LINB.west);
    \draw[-] (LINB.east) -- (CGW.west);
    %Legend
    \filldraw[palegrey, thick] (-8.3,10.3) rectangle (6.0, 13);
    \node[programmablenode] (PROG) [align=center] at (TCU|- 50,11){Programmable\\ end node};
    \node[parametrizablenode] (PARA) [right=of PROG, align=center] {Parametrizable\\ end node};
    \node[canbus] (CAN) [right=of PARA] {CAN bus};
    \node[linbus] (LIN) [right=of CAN] {LIN bus};
    \node[ethernet] (ETH) [right=of LIN, align=center] {Ethernet\\ network};
    \node[] (Legend) at (-1.1,12.5) {Legend};
    
\end{tikzpicture}
\caption{Partial block diagram of the in-vehicle networks and nodes}
\label{fig:networkoverview}
\end{figure}
\clearpage

\paragraph{Programmable end nodes}
\label{subsec:programmablenode}
As mentioned above, the programmable end nodes use a real-time operating system. From the vendor's documentation\footnote{\url{http://support.openecu.com/doc_user/openecu_user_guide_c_api/openecu_user_guide_c_api.html}, accessed 18 January 2024} we get the following system model. Tasks are implemented as single-shot functions, meaning they run until completion and don't wait. Tasks become ready due to their periodicity, which is handled by the scheduler. The scheduler is a fixed priority preemptive scheduler which itself executes every 1 millisecond. When no tasks are running a background task is scheduled until another tasks becomes ready. The documentation does not specify the behaviour when two tasks have the same priority. Numerically lower tasks have lower priority, i.e. a task with priority 1 has lower priority than a task with priority 2. A description of the tasks defined by the operating system is shown in Table~\ref{tab:priorities}. 

\begin{table}[]
    \centering
    \resizebox{\textwidth}{!}{%
    \begin{tabular}{@{}llll@{}}
    \toprule
    Priority & Task                       & Trigger            & Description                                                                                                                \\ \midrule
    14+n     & CAN messaging (event) task & 2ms periodic burst & Handles CAN messaging                                                                                                      \\
    13+n     & J1939 messaging task       & 5ms periodic       & Handles J1939 messaging                                                                                                    \\
    12+n     & PFF task                   & 10ms periodic      & Handles freeze frame processing                                                                                            \\
    11+n     & PFS task                   & 10ms periodic      & Handles non-volatile filesystem                                                                                            \\
    10+n     & PISO messaging task        & 5ms periodic       & Handles ISO-15765 messaging                                                                                                \\
    9+n      & PDG messaging task         & 10ms periodic      & Handles ISO15765 based diagnostics messaging                                                                               \\
    6..6+n   & Application tasks          & Angular            & Handles application processing                                                                                             \\
    5        & PDTC task                  & 100ms              & Handles diagnostic trouble code processing                                                                                 \\
    4        & CAN messaging task         & 2ms periodic burst & Handles CAN messaging                                                                                                      \\
    2        & Watchdog task              & 200ms periodic     & Handles the processor watchdog                                                                                             \\
    1        & CCP task                   & 5ms periodic       & \begin{tabular}[c]{@{}l@{}}Handles the CAN Calibration Protocol messaging to \\ and from the calibration tool\end{tabular}
    \end{tabular}%
    }
    \caption{Real-time operating system and application tasks for OpenECU M560 and M130 as defined by OpenECU documentation}
    \label{tab:priorities}
\end{table}

Unfortunately the documentation is a bit unclear in the specification of the tasks, two different CAN tasks with two different priorities are specified, the CAN tasks become ready based on an event and then executes for a number of iterations without specifying the event or the number of iterations. In one place the application tasks are defined to be triggered based on the crank or engine position, this is contradictory to other parts of the documentation where the trigger is defined as a periodic value given by the user. The following assumptions are made: the low priority CAN task is responsible for transmitting messages while the high priority task is responsible for reading messages. The reason being that we prefer copying received messages from the hardware peripheral's buffer to a software buffer than transmitting a message and missing a CAN message because the receive buffers were full. The event triggering the receive task is assumed to be the reception of a CAN message in a hardware buffer. Reason being that microcontrollers often have an interrupt for this event. The number of iterations for the receive task is assumed to be 1, the task copies all available messages at that time from the hardware buffer into a software buffer. This is a simple implementation of an interrupt, more complex behaviour is possible but unlikely given the lack of documentation. The CAN transmit task is further defined in the CAN section of the documentation. It is assumed that the ECU has two hardware buffers for each CAN bus, if the application transmits a CAN frame, it first tries the hardware buffer, bypassing the CAN transmit task. If the buffer is full it is stored in the transmit task's software buffer, which triggers the transmit task to become ready. Once the task is scheduled it will put the highest priority buffered frame into arbitration for each CAN bus. If there are more buffered messages the task will be rescheduled after a 2ms cooldown. The documentation states that this is done to "lessen the load experienced by the CAN bus". Finally, the application tasks are assumed to be periodic instead of relative to the engine position.

From a programmers point of view the software is split up in several logical nodes, implementing a related set of functions. E.g the \textit{braking system manager} or the \textit{lighting manager}. The logical nodes can be further split into one or more \textit{runnables}, the runnable is the smallest software component that is subject to scheduling by the RTOS. This allows to split a logical node in several runnables which can be scheduled independently or deployed on different physical nodes. The software implementing the required functions is decoupled from the physical node it is deployed on, allowing functionality to be moved around as necessary.

Runnables communicate with each other through signals, a signal represents a sample of some system state variable, named data dictionaries by OpenECU. The system state variables can represent a physical value or some abstract system state. A data dictionary is produced by a single runnable, but can be consumed by multiple runnables. For each logical node a set of deployment files describe which runnables exist, which data dictionaries are consumed and produced by each runnable, at what rate the runnables are scheduled and on which \textit{programmable end nodes} they are deployed. Figure~\ref{fig:logical_example} shows an example of two logical nodes, each consisting of several runnables. The \textit{Brake pedal} runnable of the \textit{Braking System} logical produces a \textit{brake pedal applied} data dictionary which is consumed by the \textit{Rear light} runnable of the \textit{Lighting Manager} logical.

\begin{figure}[htb]
    \centering
    \resizebox{0.95\textwidth}{!}{%
        \tikzfig{logical_example}
    }
 \caption{Example of two logical components consisting of several runnables communicating through data dictionaries}
\label{fig:logical_example}
\end{figure}

The deployment files are used when building binaries for the programmable end nodes to automatically generate code to bridge the required data dictionaries between the runnables. First let's consider a pair of runnables consuming/producing the same data dictionary which are deployed on the same programmable end node. One runnable generates a data dictionary while the other is a consumer of that data dictionary. Because the runnables are deployed on the same programmable end nodes no network traffic is required. In this case the data dictionary can be modelled as a shared global variable residing in the programmable end nodes memory which can be accessed in a thread-safe way by both runnables.

If the producing and consuming runnables are deployed on different \textit{programmable end nodes} which are directly connected to each other, the data dictionary has to be transmitted on the network connecting them. The data dictionaries are transmitted at a fixed rate by a \textit{CAN transmit} runnable. The runnable reads at once all the data dictionaries from the shared global variables, packs them in the predefined CAN frames and schedules them for transmission by the real-time operating system at once. A \textit{CAN receive} runnable is scheduled at a fixed rate which retrieves the last received instance of each registered CAN frame from the real-time operating system and unpacks the frame into the shared global variables at once. If a frame was not received between two executions the real-time operating system returns the last received instance together with an error code indicating that it is an old frame. As the scheduler is preemptive, both these processes can be preempted by higher priority tasks, care should be taken when setting priorities of the runnables. For efficiency reasons data dictionaries that have the same \textit{programmable end node} as destinations are packed in the same CAN message. The definition of the CAN messages is generated automatically at build time and can vary per build. The packing and CAN IDs are picked in a non-deterministic way without an optimization criterion, except for minimizing the number of message definitions. Figure~\ref{fig:physical_example} shows an example deployment of two logical nodes on two physical nodes.

\begin{figure}[htb]
    \centering
    \resizebox{\textwidth}{!}{%
        \tikzfig{deployment_example}
    }
 \caption{Example of two logical components consisting of several runnables communicating through data dictionaries}
\label{fig:physical_example}
\end{figure}

Lastly, if the producing and consuming runnables are deployed on different \textit{programmable end nodes} which are not directly connected to each other, the data dictionary must be bridged across networks by the \textit{Central Gateway}. Bridging is implemented by the \textit{CAN transmit} and \textit{CAN receive} tasks and is also generated automatically. The data dictionaries that need to be bridged but are not used by the \textit{Central Gateway} are stored similarly to data dictionaries which are used, but other runnables can't read or write them. This means that some extra delay is introduced in the transmission of bridged data dictionaries by the scheduling of the receive and transmit runnables in the \textit{Central Gateway}.

\todo{Mapping of runnables to scheduled tasks in reality}
\todo{Task priorities}
\todo{Describe RTOS CAN receive/transmit}
\newpage
\subsection{Modelling the network}
\label{subsec:moddeling}
In this section we will describe how the Lightyear 0's embedded system as defined in Section~\ref{subsec:network} is modelled in OMNeT++. We also mention the assumptions made and the differences between the model and the real embedded system. The FiCo4OMNeT CAN models where used to avoid implementing all the necessary details of a CAN bus and devices. Changes where made to FiCo4OMNeT such that it works with \omnet 6.0.2, these changes are mostly implementation details, but also contain some bug fixes such as an incorrect message size for normal identifiers when extended identifiers are allowed in the network.

\paragraph{Programmable end nodes} OMNeT++ differentiates between simple modules and compound modules. Simple modules can be seen as the lowest level of the model, they implement the behaviour by means of C++ code which acts on the events received by that module. Compound modules do not have an implementation but serve as an abstraction layer grouping submodules and the connections between those submodules inside a named abstraction.

The programmable end nodes are implemented as a compound module. A generic compound module \textit{LyPhysicalNode} aggregates the internals of a programmable end node. Specializations of the \textit{LyPhysicalNode} implement the final specific programmable end node, e.g. the \textit{VCUNode} compound module represents the Vehicle Control Unit by filling in details like the number of data dictionaries and their names, the number of runnables with the names and priorities and the read/write relationship between runnables and data dictionaries of the real world Vehicle Control Unit. The specializations are generated from the source code such that they closely resemble the real world implementation. 

A \textit{LyPhysicalNode} consists of at least the following modules: the scheduler, one CAN source application, one CAN sink application, one up to four CAN devices. These modules represent the OpenECU RTOS as specified in section~\ref{subsec:programmablenode}. Their behaviour is modelled according to the documentation and the assumptions mentioned in section~\ref{subsec:programmablenode}. A graphical representation of the compound module as presented by \omnet can be found in Figure~\ref{fig:lyphysicalnode}

\begin{figure}[htb]
    \centering
    \includegraphics[width=0.5\textwidth]{images/LyPhysicalNode_example.png}
    \caption{Example LyPhysicalNode compound module}
    \label{fig:lyphysicalnode}
\end{figure}

For simplicity, we have only modelled the two CAN messaging tasks, the application background task from Table~\ref{tab:priorities} and the runnables implemented by Lightyear. The documentation did not specify the scheduler's behaviour when two or more tasks of the same priority level are eligible for execution, we chose to take a random task from the list of the highest priority ready and paused tasks. In reality there could be some implicit ordering as the tasks are listed in some datastructure such as an array or list through which the scheduler iterates. 

The CAN device is a module imported from the FiCo4OMNeT model~\cite{meyer2019simulation}. It takes care of CAN message arbitration by implementing a priority based schedule. The device has an infinite amount of transmit buffers, which is not realistic. The source application module limits the number of messages buffered for transmission by the CAN device to a constant called \textit{hardwareBufferSize} which can be modified in the experiment setup. If more than two CAN messages are scheduled for transmission by the runnables the CAN source task will store them in a software buffer following the priority queue mechanism. If a frame with a specific ID is retransmitted by a runnable before it was moved from the software buffer to the hardware buffer it will be overwritten as is defined in the documentation. All messages transmitted by runnables pass through the single CAN source application such that there is a single module responsible for maintaining the CAN bus invariants as defined in the OpenECU documentation. \todo{explain effects of this no realism on the validaity of a simulation.}

The CAN device implemented by FiCo4OMNeT has no receive buffers and will immediately transfer a received message to the sink application. This is not realistic behaviour, often a CAN peripheral will have one or more hardware receive buffers. And emits an interrupt, signalling that a CAN frame is available. For this reason the reception of a CAN message does not require any scheduling of the CAN sink application in our model. Once one or more messages are buffered, the sink app will change its state from Blocked to Ready, signalling to the scheduler that it needs to be executed. Upon execution the sink application will move the received frames to its software buffer. Internally the sink application keeps track of the last received instance of each registered CAN message in a software buffer, consistent with the OpenECU documentation. Runnables can request the last received instance of a CAN message from the sink application. Such a request does not cause the sink application to be scheduled but is seen as run-time of the requesting application.

The runnables, accidentally named logicals in the model, are tasks which can be scheduled independently of each other. It is assumed that each runnable is periodic, thus they have a configurable priority, period and execution time. As mentioned earlier, runnables can request the latest received CAN message from the sink application. This request takes no time to complete and does not cause the sink application to be scheduled, the required execution time for such a request is assumed to be part of the runnable's execution time. It is assumed that all CAN frames are requested at the start of a logical's execution. Runnables can also request the transmission of a frame on a specific CAN bus. This is done by sending a CAN frame to the source application, the exact behaviour of the source application is described above. All transmissions are performed at the end of the runnable's execution. In reality read and send requests can occur throughout the runnable's execution, but for simplicity this model was chosen. Similarly, data dictionaries can be read or written by a runnable. All reads occur at the start of execution and take zero time to execute. All writes occur at the end of execution and also take zero time to execute.

\todo{beschrijf probleem van synchronisatie van data dictionaries in netwerk en dat we daarom de age bijhouden.}
Upon creation of a data dictionary sample a write-count is recorded inside the sample and an observer node records the generation time associated with that write count. Each time the data dictionary is read the observer node receives a signal containing the write-count of the read sample. Upon which the observer node calculates the age of the sample when it was read and stores that information for further analysis. To facilitate the transmission of data dictionaries between nodes Lightyear implemented two runnables responsible for receiving and transmitting CAN messages on each programmable end node. For messages received and transmitted by these runnables a one-to-one mapping between data dictionaries and CAN frame payload is known. The write-count can thus be reliably registered and forwarded alongside the CAN frame for further analysis when it is read by runnables in subsequent parametrizable end nodes. Unfortunately such a mapping is not available for CAN messages going to parametrizable end nodes. In case such a mapping is available the model will automatically transmit relevant data dictionary meta-data alongside the CAN frame.
\todo{wat zijn consequencies}

\paragraph{Parametrizable end nodes} As described in Section~\ref{subsec:network} the remaining nodes in the network are a mix of third party devices and fixed function devices created by Lightyear. The inner workings of these parametrizable end nodes is (mostly) unknown and only a specification of the CAN traffic is known. From the source code of programmable end node runnables interfacing with these nodes we deduce that most of the traffic is periodic. Unfortunately no structured data except the source code exists which describes which messages are periodic and which are aperiodic. As it seems that the majority of the traffic is periodic our model only implements periodic transmission of messages by the parametrizable end nodes. FiCo4OMNeT contains a module representing a CAN node which periodically transmits CAN messages. The module can also be configured to receive specific messages. The FiCo4OMNeT module is used to implement the parametrizable end node without further modifications.

\paragraph{CAN network} All CAN busses have been modelled using the CAN bus model provided by FiCo4OMNeT. This model takes care of details such as bit stuffing, error frame injection and handling and arbitration. Some changes were needed to make it compatible with the latest version of \omnet. These changes are publicly available\footnote{\url{https://github.com/StephanOostveen/FiCo4OMNeT/tree/feature/omnet6}} and have been sent to the package maintainers. 

\subsection{Model generation from source code}
\todo{Call graph generation + why this works for LY}
\todo{Graph to read/write sheet with dd details as size etc}
\todo{Omnet++ codegen of logical/dds and connections}
\todo{DBC to OMNeT++ data}