\section{Discussion}
\label{sec:discussion}
\todo{reflect hoe moeilijk/makkelijk het was om benchmark getallen naar boven te krijgen en hoe je dit op architectuur niveau makkelijker zou kunnen maken om te analyseren.}

Runnables with the same period deployed on a certain programmable end node have been grouped in a single task which is scheduled by the operating system, see Listing~\ref{lst:rtos_task} for an example. The runnables inside a task have been given a specific order in which they are executed. The reason for this deviation in the initial model or the reason for the specific order in which runnables are executed is unknown. 

\lstinputlisting[caption=Example scheduled task, label={lst:rtos_task}, language=C, firstline=1,frame=single,numbers=left,keywordstyle=\color{blue}]{example_task.c}

\todo{discuss the effect of this deviation for the results -> probably not such an issue as the runnables in a task have the same, but order in which they occur is semi random}

\todo{discuss missing xcp/ccp data streams and that they have negative effect on system validation because it changes the CAN bus timings in theory}

\todo{Discuss inverter data, benefits of synchronization}

\todo{Discuss more accurate bit stuffing can be implemented by accurately stuffing the message by sending random data in the packet that can be stuffed.}

\todo{drawback, CAN RX/TX is event based, not each ecu is strictly rate monotonic, utilization is incorrect }